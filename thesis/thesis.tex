%\documentclass[PhD]{iitmdiss}
%\documentclass[MS]{iitmdiss}
%\documentclass[MTech]{iitmdiss}
\documentclass[BTech]{iitmdiss}
%\usepackage{times}
 \usepackage{t1enc}

\usepackage{graphicx}
\usepackage{epstopdf}
\usepackage[hypertex]{hyperref} % hyperlinks for references.
\usepackage{amsmath} % easier math formulae, align, subequations \ldots

\begin{document}


%%%%%%%%%%%%%%%%%%%%%%%%%%%%%%%%%%%%%%%%%%%%%%%%%%%%%%%%%%%%%%%%%%%%%%
% Title page

\title{Block Sorting Heuristics}

\author{Venketep Prasad M C}

\date{May 2018}
\department{COMPUTER SCIENCE AND ENGINEERING}

%\nocite{*}
\maketitle

%%%%%%%%%%%%%%%%%%%%%%%%%%%%%%%%%%%%%%%%%%%%%%%%%%%%%%%%%%%%%%%%%%%%%%
% Certificate
\certificate

\vspace*{0.5in}

\noindent This is to certify that the thesis titled {\bf BLOCK SORTING HEURISTICS}, submitted by {\bf VENKETEP PRASAD M C}, 
  to the Indian Institute of Technology, Madras, for
the award of the degree of {\bf B.tech}, is a bona fide
record of the research work done by him under our supervision.  The
contents of this thesis, in full or in parts, have not been submitted
to any other Institute or University for the award of any degree or
diploma.

\vspace*{1.5in}

\begin{singlespacing}
\hspace*{-0.25in}
\parbox{2.5in}{
\noindent {\bf Dr. N S Narayanaswamy} \\
\noindent Research Guide \\ 
\noindent Professor \\
\noindent Dept. of CSE\\
\noindent IIT Madras, 600036 \\
} 
\hspace*{1.0in} 
%\parbox{2.5in}{
%\noindent {\bf Prof.~S.~C.~Rajan} \\
%\noindent Research Guide \\ 
%\noindent Assistant Professor \\
%\noindent Dept.  of  Aerospace Engineering\\
%\noindent IIT-Madras, 600 036 \\
%}  
\end{singlespacing}
\vspace*{0.25in}
\noindent Place: Chennai\\
Date: 30th April 2018 


%%%%%%%%%%%%%%%%%%%%%%%%%%%%%%%%%%%%%%%%%%%%%%%%%%%%%%%%%%%%%%%%%%%%%%
% Acknowledgements
\acknowledgements

Thanks to all those who made \TeX\ and \LaTeX\ what it is today.

%%%%%%%%%%%%%%%%%%%%%%%%%%%%%%%%%%%%%%%%%%%%%%%%%%%%%%%%%%%%%%%%%%%%%%
% Abstract

\abstract

\noindent KEYWORDS: \hspace*{0.5em} \parbox[t]{4.4in}{\LaTeX ; Thesis;
  Style files; Format.}

\vspace*{24pt}

\noindent A \LaTeX\ class along with a simple template thesis are
provided here.  These can be used to easily write a thesis suitable
for submission at IIT-Madras.  The class provides options to format
PhD, MS, M.Tech.\ and B.Tech.\ thesis.  It also allows one to write a
synopsis using the same class file.  Also provided is a BIB\TeX\ style
file that formats all bibliography entries as per the IITM format.

The formatting is as (as far as the author is aware) per the current
institute guidelines.

\pagebreak

%%%%%%%%%%%%%%%%%%%%%%%%%%%%%%%%%%%%%%%%%%%%%%%%%%%%%%%%%%%%%%%%%
% Table of contents etc.

\begin{singlespace}
\tableofcontents
\thispagestyle{empty}

\listoftables
\addcontentsline{toc}{chapter}{LIST OF TABLES}
\listoffigures
\addcontentsline{toc}{chapter}{LIST OF FIGURES}
\end{singlespace}


%%%%%%%%%%%%%%%%%%%%%%%%%%%%%%%%%%%%%%%%%%%%%%%%%%%%%%%%%%%%%%%%%%%%%%
% Abbreviations
\abbreviations

\noindent 
\begin{tabbing}
xxxxxxxxxxx \= xxxxxxxxxxxxxxxxxxxxxxxxxxxxxxxxxxxxxxxxxxxxxxxx \kill
\textbf{IITM}   \> Indian Institute of Technology, Madras \\
\textbf{RTFM} \> Read the Fine Manual \\
\end{tabbing}

\pagebreak

%%%%%%%%%%%%%%%%%%%%%%%%%%%%%%%%%%%%%%%%%%%%%%%%%%%%%%%%%%%%%%%%%%%%%%
% Notation

\chapter*{\centerline{NOTATION}}
\addcontentsline{toc}{chapter}{NOTATION}

\begin{singlespace}
\begin{tabbing}
xxxxxxxxxxx \= xxxxxxxxxxxxxxxxxxxxxxxxxxxxxxxxxxxxxxxxxxxxxxxx \kill
\textbf{$r$}  \> Radius, $m$ \\
\textbf{$\alpha$}  \> Angle of thesis in degrees \\
\textbf{$\beta$}   \> Flight path in degrees \\
\end{tabbing}
\end{singlespace}

\pagebreak
\clearpage

% The main text will follow from this point so set the page numbering
% to arabic from here on.
\pagenumbering{arabic}


%%%%%%%%%%%%%%%%%%%%%%%%%%%%%%%%%%%%%%%%%%%%%%%%%%
% Introduction.

\chapter{INTRODUCTION}
\label{chap:intro}
Mathmatics, Computer science, Informatics and Machine learning has lot of application in the field of Bio Informatics. Specific field like genome sequence matching has a lot of interesting problems for computer scientists. A Genome is a whole set of DNA. DNAs contain chromosomes which forms blocks of genome. Each chromosome contains a set of genes which are the fundamental reason for heredity. The order in which genes appear in a chromosome is responsible for its functionality, So different order implies different functionality. During molecular evolution there involves rearrangement of genes. Over the years these rearrangements are responsible for different behaviours in different species. There are some set of events which occur during a rearragement. Reversal, Translocation, Block or Strip moves are few such events. \\
A Chromosome is represented as a sequence of genes. Each gene is given a mapping in number space hence a chromosome with 10 genes might look like
$$4\text{ }10\text{ }8\text{ }6\text{ }5\text{ }3\text{ }1\text{ }2\text{ }7\text{ }9$$
Initially in molecular biology experts were interested in local mutations and hence concentrating on genes. Later they shifted to global level gene rearragements in a chromosome. If two species contains different chromosomes $C_1$ and $C_2$, but both of them constitute same set of genes, then the similarity between the two chromosomes is defined as minimum number of primitive steps from one chromosome to other. Whenever two chromosomes were compared, one of them will be taken as a base or identity permutation and the other chromosome will be altered using primitive steps to obtain identity permutation. The primitive steps are the possible rearrangements which are Reversals, Transpositions, Block moves. The minimum number of moves to reach identity permutaition is denoted as $D(\pi)$. This distance is a measure of similarity between the two permutations. Lesser distance implies greater similarity. Finding minimum event set resulting in an identity permutation is NP-hard in case of Reversals and Block moves, whereas for transpositions the complexity remains open. For these hard problems we are in search of approximation algorithms which run in polynomial time. 
\section{Sorting by Reversal}
An event in Reversal of a permutation $\pi$ = $\pi_1\pi_2....\pi_n$ is to choose a substring and reverse it. So if we reverse the substring $\pi_i...\pi_j$, then after the reversal event, $\pi = \pi_1...\pi_{i-1}\pi_j....\pi_i\pi_{j+1}\pi_n$. For example, Consider permutation 1 8 2 3 \textbf{5 6 4} 7 9 10 $\rightarrow$ 1 8 2 3 4 \textbf{6 5} 7 9 10 $\rightarrow$ 1 8 \textbf{2 3 4 5 6 7} 9 10 $\rightarrow$ 1 \textbf{8 7 6 5 4 3 2} 9 10 $\rightarrow$ 1 2 3 4 5 6 7 8 9 10. In each event the substring highlighted is reversed. Hence the distance is 4. 

\section{Sorting by Transpositions}
An event in Transposition of a permutation $\pi$ = $\pi_1\pi_2....\pi_n$ is to choose two adjacent substring and swap them. Let say in an event we transpose substring $[\pi_i,\pi_{j-1}]$ and $[\pi_j,\pi_k]$, after the event permutation $\pi = \pi_1...\pi_{i-1}\pi_j....\pi_k\pi_i....\pi_{j-1}\pi_{k+1}\pi_n$. For example, Consider permutation 7 8 1 2 \textbf{4 5 6} \textit{3} 9 $\rightarrow$ \textbf{7 8} \textit{1 2 3 4 5 6} 9 $\rightarrow$ 1 2 3 4 5 6 7 8 9. In each event the substring highlighted is swapped with the substring in italic. Hence the distance is 3.\\
Table~\ref{tab:sample} shows the current status of different sorting primitives.

\begin{table}[htbp]
  \caption{Current status of various Sorting primitives}
  \begin{center}
  \begin{tabular}[c]{|l|c|c|} \hline
    \textbf{Primitive} & \textbf{Complexity} & \textbf{Best Approximation}\\ \hline
    Reversals & NP-hard & 1.375  \\ \hline
    Transposition & Open & 1.375 \\ \hline
    Block sort & NP-hard & 2 \\ \hline
  \end{tabular}
  \label{tab:sample}
  \end{center}
\end{table}


\begin{figure}[htpb]
  \begin{center}
    \resizebox{50mm}{!} {\includegraphics *{iitm.eps}}
    \resizebox{50mm}{!} {\includegraphics *{iitm.eps}}
    \caption {Two IITM logos in a row.  This is also an
      illustration of a very long figure caption that wraps around two
      two lines.  Notice that the caption is single-spaced.}
  \label{fig:iitm}
  \end{center}
\end{figure}


\section{Bibliography with BIB\TeX}

I strongly recommend that you use BIB\TeX\ to automatically generate
your bibliography.  It makes managing your references much easier.  It
is an excellent way to organize your references and reuse them.  You
can use one set of entries for your references and cite them in your
thesis, papers and reports.  If you haven't used it anytime before
please invest some time learning how to use it.  

I've included a simple example BIB\TeX\ file along in this directory
called \verb+refs.bib+.  The \verb+iitmdiss.cls+ class package which
is used in this thesis and for the synopsis uses the \verb+natbib+
package to format the references along with a customized bibliography
style provided as the \verb+iitm.bst+ file in the directory containing
\verb+thesis.tex+.  Documentation for the \verb+natbib+ package should
be available in your distribution of \LaTeX.  Basically, to cite the
author along with the author name and year use \verb+\cite{key}+ where
\verb+key+ is the citation key for your bibliography entry.  You can
also use \verb+\citet{key}+ to get the same effect.  To make the
citation without the author name in the main text but inside the
parenthesis use \verb+\citep{key}+.  The following paragraph shows how
citations can be used in text effectively.

More information on BIB\TeX\ is available in the book by
\cite{lamport:86}.  There are many
references~\citep{lamport:86,prabhu:xx} that explain how to use
BIB\TeX.  Read the \verb+natbib+ package documentation for more
details on how to cite things differently.

Here are other references for example.  \citet{viz:mayavi} presents a
Python based visualization system called MayaVi in a conference paper.
\citet{pan:pr:flat-fst} illustrates a journal article with multiple
authors.  Python~\citep{py:python} is a programming language and is
cited here to show how to cite something that is best identified with
a URL.

\section{Other useful \LaTeX\ packages}

The following packages might be useful when writing your thesis.

\begin{itemize}  
\item It is very useful to include line numbers in your document.
  That way, it is very easy for people to suggest corrections to your
  text.  I recommend the use of the \texttt{lineno} package for this
  purpose.  This is not a standard package but can be obtained on the
  internet.  The directory containing this file should contain a
  lineno directory that includes the package along with documentation
  for it.

\item The \texttt{listings} package should be available with your
  distribution of \LaTeX.  This package is very useful when one needs
  to list source code or pseudo-code.

\item For special figure captions the \texttt{ccaption} package may be
  useful.  This is specially useful if one has a figure that spans
  more than two pages and you need to use the same figure number.

\item The notation page can be entered manually or automatically
  generated using the \texttt{nomencl} package.

\end{itemize}

More details on how to use these specific packages are available along
with the documentation of the respective packages.

%%%%%%%%%%%%%%%%%%%%%%%%%%%%%%%%%%%%%%%%%%%%%%%%%%%%%%%%%%%%
% Appendices.

\appendix

\chapter{A SAMPLE APPENDIX}

Just put in text as you would into any chapter with sections and
whatnot.  Thats the end of it.

%%%%%%%%%%%%%%%%%%%%%%%%%%%%%%%%%%%%%%%%%%%%%%%%%%%%%%%%%%%%
% Bibliography.

\begin{singlespace}
  \bibliography{refs}
\end{singlespace}


%%%%%%%%%%%%%%%%%%%%%%%%%%%%%%%%%%%%%%%%%%%%%%%%%%%%%%%%%%%%
% List of papers

\listofpapers

\begin{enumerate}  
\item Authors....  \newblock
 Title...
  \newblock {\em Journal}, Volume,
  Page, (year).
\end{enumerate}  

\end{document}
